\newglossaryentry{Malware}
{
    name=Malware,
    description={O software "malicioso" es todo aquél programa o código que pretende (de forma intencionada) causar daños y/o sacar beneficios de un sistema}
}

\newglossaryentry{cloud}
{
    name=Cloud,
    description={La computación en la nube o \textit{cloud} (del inglés cloud computing), conocida también como servicios en la nube, informática en la nube, nube de cómputo o simplemente «la nube», es un paradigma que permite ofrecer servicios de computación a través de una red, que usualmente es internet.}
}

\newglossaryentry{Ransomware}
{
    name=Ransomware,
    description={Es un tipo de malware que pretende hacer públicos o inaccesibles (por medio de encriptación, por ejemplo) los datos de la víctima hasta que esta pague un "rescate".}
}


\newglossaryentry{Reverse engineering}
{
    name={ingeniería inversa},
    description={Es una técnica que consiste en intentar obtener información sobre cómo está hecho o cómo funciona un producto a partir del propio producto en si. Intentando adivinar cómo funciona por medio de su uso y de pruebas que corroboren las suposiciones que se hagan.}
}

\newglossaryentry{obfuscation}{
    name={obfuscation},
    description={Ofuscación, ocultación, anonimato. Es el acto de evitar ser descubierto mientras realizas un ataque o una auditoría. Bien eliminando los rastros que puedas dejar u ocultándolos por medio de falsos rastros que escondan los tuyos.}
}

\newglossaryentry{network sniffing}
{
    name=network sniffing,
    description={Es la acción de "atender" al tráfico indiscriminado que circula por una red como si se fueran todos los posibles destinatarios con el objetivo de obtener una información que no fuera destinada a nosotros}
}

\newglossaryentry{Rolling Release}
{
    name=network sniffing,
    description={Es un tipo de distribución de Software en el que las actualizaciones son continuas en lugar de depender de un versionado discreto. Los cambios se van añadiendo de forma incremental conforme van siendo disponibles en lugar de ir emitiendo nuevas versiones con todos los cambios desde la anterior.}
}

\newglossaryentry{compilance}
{
    name=compilance,
    description={'el cumplimiento' de las normativas o leyes referentes a la seguridad de los datos que una empresa pueda almacenar o gestionar}
}

\newglossaryentry{vagrant}
{
    name=Vagrant,
    description={'una herramienta diseñada para el despliegue y configuración de entornos de máquinas virtuales (utilizando diversos proveedores como virtualbox, qemu, aws, etc...}
}

\newacronym{ddos}{DDOS}{Distributed Denial Of Service}

\newacronym{aws}{AWS}{Amazon Web Service}


\newacronym{iot}{IOT}{The Internet of Things}

\newglossaryentry{OpenSource}
{
    name=OpenSource,
    description={OpenSource o código abierto es un tipo de software liberado con una licencia que asegura el derecho de los usuarios a usar, estudiar, cambiar y distribuir el mismo con cualquier propósito.}
}

\newglossaryentry{Pull Request}
{
    name=OpenSource,
    description={Una Pull Request es la acción de validar un código que se va a mergear de una rama a otra. Por ejemplo, de una rama de desarrollo en un Fork de un proyecto a una rama oficial.}
}

\newacronym{IaC}{IAC}{Infrastructure as Code}
\newacronym{CPA}{CPA}{Certified Public Accountant}


